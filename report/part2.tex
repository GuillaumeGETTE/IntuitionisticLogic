\section{Construction d'une preuve en logique intuitionniste}

\subsection{Modification de l'algorithme}

Notre algorithme d\'ebute par un pr\'etraitement des donn\'ees d'entr\'e.
Tout d'abord, l'ensemble des atomes sont renomm\'es pour permettre l'introduction de nouveaux atomes durant l'\'etablissement des ensembles $S$ et $X$. La formule est ensuite mise sous forme canonique selon les r\`egles explicit\'ees en page 625 de \ref{intuit2015} de la mani\`ere suivante : la formule est pouss\'ee dans une liste vide, puis pour chaque élément de cette liste, l'algorithme essaye d'appliquer une des r\`egles. En cas de réussite, l'\'element est d\'epil\'e et les formules g\'en\'er\'ees sont ajout\'ees en d\'ebut de liste, en cas d'\'echec l'algorithme met cet \'element de c\^ot\'e et passe aux suivants. Ce traitement g\'en\`ere de nouveaux atomes mais r\'eduit la taille des clauses, donc termine. Enfin la liste est parcourue pour s\'eparer les clauses plates (ensemble $S$) des clauses de type implication (ensemble $X$), et pour chaque clause $(a \rightarrow b) \rightarrow c$ de $X$, la clause $b \rightarrow c$ est ajout\'ee \`a $S$.

Une fois la mise sous forme canonique achev\'ee, nous passons à la phase de recherche de preuve, que nous avons rendu récursive. Cette phase se compose de deux fonctions, \textit{intuitCheck} et \textit{intuitProve} :


- flat clauses S
- implication clauses X
- assumptions A
- proof goal q
- SAT-proof p
- Intuitionistic proofs p', p2 and p3

\begin{listing}[1]{1}
procedure intuitProve (s, X , A, q)
switch satProve (s, A, q)
    case YesSAT (A', p) :
        return YesI (A', p');
    case NoSAT (M ) :
        switch intuitCheck (s, X , M )
            case True :
                return NoI (M );
            case False (M*, c, i, X*, p2):
                switch intuitProve(($M* \rightarrow c$)::s,X,A,q)
                    case YesI(A', p1) :
                        return YesI(A', p3);
                    case NoI(M'):
                        return NoI(M');
\end{listing}

avec p' :
\begin{prooftree}
\AxiomC{p}
\LeftLabel{SAT}
\UnaryInfC{$S, X, A \vdash q$}
\end{prooftree}

et p3 :
\begin{prooftree}
\AxiomC{p1}
\AxiomC{M*, S, X-{i}, i}
\AxiomC{p2}
\LeftLabel{IMPL}
\BinaryInfC{$S, X-{i}, \vdash M* \rightarrow c$}
\LeftLabel{MP}
\BinaryInfC{$S, X, A \vdash q$}
\end{prooftree}



-- SAT-solver s
-- implication clauses X
-- model M

\begin{listing}[1]{1}
procedure intuitCheck (s, X , M )
    for i ∈ X :
        let (a → b) → c = i
        if a, b, c $\notin$ M then
            switch intuitProve (s, X − {i}, M ∪ {a}, b)
                case YesI (M*,p) :
                    return False(M*, c, i, X-{i}, p)
                case NoI(M')
                return True;
\end{listing}
                
\subsection{R\'ecriture d'une preuve classique en intuitionniste}
